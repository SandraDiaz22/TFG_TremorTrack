\capitulo{2}{Objetivos del proyecto}

El objetivo principal del proyecto es desarrollar una aplicación web que ofrezca a las personas con párkinson una manera intuitiva de llevar un seguimiento de su enfermedad, permitiéndoles ver y predecir su evolución, además de permitir a los médicos gestionar la enfermedad de sus pacientes.

Para ello se han definido una serie de objetivos, entre los que se pueden distinguir objetivos marcados por los requisitos del \textit{software} y objetivos de carácter técnico:


\subsection{Objetivos marcados por los requisitos del \textit{software}}
\begin{itemize}
    \item Diseñar una aplicación intuitiva y fácil de utilizar, tanto para los pacientes como para los médicos, utilizando el \textit{framework} Flask.
    \item Crear funcionalidades en la aplicación web que permitan a los pacientes acceder a gráficas que muestren los datos medidos por el sensor y predecir datos futuros sobre su evolución.
    \item Crear funcionalidades en la aplicación web que permitan a los médicos gestionar la información de sus pacientes, añadir vídeos y datos a sus historiales y ver sus gráficas y predicciones.
    \item Diseñar una base de datos en la que almacenar la información que utiliza la aplicación.
\end{itemize}


\subsection{Objetivos de carácter técnico}
\begin{itemize}
    \item Utilizar una metodología ágil de tipo Scrum mediante la herramienta Jira para realizar un seguimiento de las actividades realizadas, así como para organizar temporalmente las tareas que vaya marcando el tutor.
    \item Realizar la documentación del proyecto completa utilizando \LaTeX{}, de forma progresiva a la evolución del mismo, mostrando a personas ajenas todos los aspectos relacionados con el proyecto realizado.
    \item Lectura y aprendizaje sobre la enfermedad de Parkinson, sus síntomas y tratamientos, así como el alcance de la enfermedad para obtener el contexto médico del proyecto.
    \item Experimentación con diferentes técnicas de \textit{machine learning}, implementadas mediante diferentes bibliotecas de Python, para realizar las predicciones.
    \item Experimentación con diferentes técnicas de realización de gráficos para mostrar los datos del sensor y predicciones.
    \item Mejorar los conocimientos sobre Python, \textit{machine learning}, bases de datos, análisis \textit{software} y dirección de proyectos que se estudian en el grado.
    \item Adquirir conocimientos sobre desarrollo web.
    \item Internacionalizar la aplicación web, permitiendo su uso a más personas.
\end{itemize}