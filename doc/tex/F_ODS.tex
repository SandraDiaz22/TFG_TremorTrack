\apendice{Anexo de sostenibilización curricular}

\section{Introducción}
En este último anexo se abordan los aspectos de sostenibilidad aplicados al Trabajo de Fin de Grado. Integrar la sostenibilidad en proyectos médicos o universitarios, además de en la vida diaria, es esencial para mitigar su impacto ambiental y mejorar la calidad de vida global.

Se ha utilizado como base de conocimientos la guía de la CRUE\footnote{\url{https://www.crue.org/wp-content/uploads/2020/02/Directrices_Sosteniblidad_Crue2012.pdf}}, que promueve una investigación y desarrollo responsables y sostenibles, ya que durante el alumnado las competencias adquiridas en términos de sostenibilidad fueron escasas.

Se van a desarrollar tres secciones: realizar una aplicación que ayuda al envejecimiento saludable, utilizar materiales reciclables y biodegradables en los sensores y emplear modelos de \textit{machine learning} y equipos \textit{hardware} eficientes.


\section{Envejecimiento saludable}
El envejecimiento es una gran preocupación en términos de sostenibilidad social y sanitaria. Desarrollar aplicaciones como la actualmente presentada puede tener un impacto significativo en la calidad de vida de las personas mayores (en este caso de los que sufren la enfermedad de Parkinson).

La utilización de sensores y la progresiva toma de vídeos de los pacientes permite realizar una monitorización continua, facilitando una detección temprana de avances en la enfermedad. Esto permite a los médicos llevar una gestión más efectiva de los tratamientos, ajustando las dosis en función de lo observado en las gráficas que proporciona la aplicación web.

Además la aplicación, al incluir un apartado en el que los pacientes pueden ver sus propios avances, proporciona independencia en la gestión de la enfermedad. Sentirse autosuficiente aporta un beneficio psicológico y emocional a las personas mayores, aumentando su autoestima y su sensación de control sobre sus propia vidas.

Por último, la monitorización efectiva puede reducir las hospitalizaciones y sus costes asociados, beneficiando tanto a los individuos y sus familias como al sistema sanitario. Además reduce los desplazamientos al centro de salud, reduciendo la huella de carbono de los combustibles y los problemas asociados al desplazamiento (pacientes con problemas severos de movilidad o que vivan muy alejados del hospital y no dispongan de medios de transporte para poder acudir a él).


\section{Materiales de los sensores}
Se desconocen los materiales utilizados actualmente para fabricar el dispositivo médico STAT-ON, que dispone de sensores para monitorizar los síntomas motores de los pacientes, desarrollado por el equipo de SENSE4CARE~\cite{sense4care}, y su posterior reciclado, pero se van a dar ideas para tratar de que sean lo más sostenibles posibles.

Para reducir su impacto ambiental, en su fabricación se debería fomentar el uso de materiales ecológicos y reciclables. Se pueden reemplazar los plásticos convencionales por polímeros biodegradables (como el ácido poliláctico, derivado del maíz o la caña de azúcar, o los polihidroxialcanoatos, producidos por microorganismos) que tardan menos en descomponerse. Utilizar metales como el acero inoxidable o el aluminio, con altas tasas de reciclabilidad, disminuiría la huella de carbono, además de ser duraderos y resistentes a la corrosión. Además, se debería tratar de evitar sustancias tóxicas y contaminantes como el plomo y el mercurio en los circuitos electrónicos o las baterías del dispositivo.

Para ayudar con su reciclado tras su vida útil se deberían diseñar los sensores para ser fácilmente desmontables, facilitando la separación de sus materiales para su reciclaje individual.


\section{\textit{Software}}
Durante el proyecto, es esencial el uso de algoritmos de \textit{machine learning} para analizar y posteriormente predecir datos de salud. Sin embargo, tanto el servidor como los modelos de \textit{machine learning} utilizados implican un consumo de energía significativo, que debe ser considerado en términos de sostenibilidad.

\subsection{Consumo de energía del servidor}
El proyecto se aloja en un servidor proporcionado por la UBU. El funcionamiento continuo de dicho servidor, necesario para soportar la infraestructura del proyecto actual y del resto de proyectos que aloje, conlleva un consumo energético constante. Para tratar de minimizar este consumo, sería interesante analizar el funcionamiento del equipo. Es crucial seleccionar \textit{hardware} eficiente, que ofrezca un alto rendimiento con bajo consumo. Alguna recomendación sería utilizar discos sólidos SSD en lugar de los tradicionales discos duros HDD, módulos de memoria DDR4 o DDR5, procesadores específicos para servidores o sistemas de enfriamiento eficientes como la refrigeración líquida.

\subsection{Consumo de energía de los modelos de \textit{machine learning}}
Los modelos grandes y complejos, como los utilizados para entrenar a sistemas como ChatGPT, consumen tanta energía como una ciudad mediana. Esto no solo implica un alto coste económico, sino también un gran impacto ambiental debido a la cantidad de energía que requieren y sus emisiones asociadas.

Emplear modelos más simples y con algoritmos optimizados reduce el consumo de energía, ayudando a mitigar el impacto ambiental del proyecto. Es importante considerar un equilibrio entre la precisión de los modelos y su eficiencia energética, promoviendo la sostenibilidad en el contexto actual del preocupante aumento del calentamiento global.