\capitulo{6}{Trabajos relacionados}

En este apartado se comentarán trabajos y proyectos realizados dentro del ámbito del proyecto en curso, lo cual proporcionará contexto sobre la situación actual en el mercado, permitiendo explicar la contribución específica de este proyecto.

Se abordarán los siguientes puntos: 
\begin{itemize}
\item Ámbitos del proyecto.
\item Aplicaciones relacionadas con la gestión de datos médicos y el seguimiento de pacientes.
\item Aprendizaje automático aplicado a la medicina.
\item Otros Trabajos de Fin de Grado (TFGs) relacionados con el párkinson.
\item Justificación de la necesidad del proyecto.
\end{itemize}



\section{Ámbitos del proyecto}
El proyecto fusiona el campo de la informática y la medicina, relacionando la creación de aplicaciones web y el uso de \textit{machine learning} con necesidades médicas, específicamente centradas en la enfermedad de Parkinson.

El campo de la medicina se caracteriza por su privatización, aunque se dispone de una amplia variedad de artículos accesibles en fuentes fiables como bibliotecas universitarias, centros de investigación o revistas científicas.

Algunas de estas fuentes incluyen la Organización Mundial de la Salud\footnote{\url{https://www.who.int/es/news-room/fact-sheets/detail/parkinson-disease}}, la PMC\footnote{\url{https://www.ncbi.nlm.nih.gov/pmc/?term=parkinson+disease}}, que ofrece revistas biomédicas y artículos de la Biblioteca Nacional de EE.UU., revistas digitales como Journal of Parkinson’s Disease\footnote{\url{https://www.journalofparkinsonsdisease.com/}}, que publica investigaciones relacionadas con la enfermedad de Parkinson o la propia biblioteca de la UBU\footnote{\url{https://ubucat.ubu.es/discovery/search?vid=34BUC_UBU:VU1}}.

El campo de la informática y la programación web está en constante evolución, con aplicaciones web desempeñando un papel crucial en la vida cotidiana, incluyendo el trabajo, los estudios, el ocio (juegos, redes sociales o comercio electrónico) o la atención médica. Existen multitud de aplicaciones web en el mercado, pero el apartado se centrará en aquellas que permiten visualizar datos médicos a pacientes y personal sanitario. 



\section{Aplicaciones médicas}
Es destacable resaltar como tras la época de restricciones debidas a la pandemia, se fomentó el uso de aplicaciones médicas para recibir atención sanitaria, así como para consultar registros médicos o gestionar citas de forma remota.

Se presentan a continuación algunas de estas aplicaciones:
\begin{itemize}
    \item 75health\footnote{\url{https://www.75health.com/}}: permite a los médicos gestionar historias clínicas y emitir recetas a sus pacientes. Por su parte, los pacientes tienen acceso a sus datos médicos y pueden contactar con su especialista. El problema de esta aplicación es que solo está disponible en Estados Unidos e India, además de que, si se quiere disponer de todas sus funcionalidades, se debe obtener una versión de pago, lo que la hace poco accesible.
    
    \item EpicCare EMR\footnote{\url{https://www.emrsystems.net/epic-emr-software/}}: la aplicación está dirigida a profesionales sanitarios que trabajan en grandes instituciones médicas, proporcionándoles herramientas para administrar la documentación y optimizar su trabajo. A su vez, ofrece a los pacientes la posibilidad de comunicarse con sus médicos y programar citas, incluso mediante videoconferencias, así como acceder a su información e historial médico. Sin embargo, no cuenta con una versión gratuita, aunque se puede solicitar una demostración. Asimismo, está disponible únicamente en inglés.
    
    \item MyChart\footnote{\url{https://www.mychart.org/Features}}: otra aplicación similar a las mencionadas anteriormente. Permite la visualización de resultados de laboratorio, historial médico, programación de citas y comunicación con el médico, entre otras funcionalidades. Aunque, nuevamente, no se encuentra disponible en España.
    
    \item PatientsLikeMe\footnote{\url{https://www.patientslikeme.com/}}: en este caso se trata de una aplicación web destinada a pacientes con enfermedades crónicas como es el párkinson. Se utiliza mayoritariamente para compartir experiencias y obtener apoyo del resto de usuarios. Además, permite el seguimiento de la enfermedad y los tratamientos. Por desgracia, esta aplicación está igualmente solo disponible en inglés.
\end{itemize}



\section{Aprendizaje automático aplicado a la medicina}
El campo del \textit{machine learning} o aprendizaje automático aplicado a la medicina es muy innovador. Existen proyectos que analizan grandes cantidades de datos médicos para realizar predicciones que ayudan con diagnósticos y tratamientos.

Un ejemplo es un proyecto publicado en la base de datos científica de la Universidad de La Rioja, donde se analizaron 13 métodos de \textit{machine learning} para comprobar cuál era más útil en la predicción de la diabetes mellitus tipo 2 en pacientes mayores de edad~\cite{prediccionDiabetes}. Los resultados indicaron que el modelo LightGBM demostró mejores resultados de precisión, sensibilidad, tasa de clasificación errónea, etc. Dado que la diabetes es una de las diez primeras causas de mortalidad entre la población adulta, la capacidad de realizar diagnósticos tempranos de forma rápida gracias a la inteligencia artificial puede ser crucial para prevenir futuras complicaciones.

Otro buen proyecto relacionado con el uso del aprendizaje automático para predecir enfermedades es el Trabajo de Fin de Grado entregado por Javier Pérez Córdova en la Universidad de Cataluña~\cite{perez2021tecnicas}. En este estudio se tratan de aplicar modelos de \textit{machine learning} basados en árboles de decisión para detectar el cáncer de mama de forma menos invasiva que mediante una mamografía convencional. Se evaluaron diversos métodos, dando lugar a resultados más y menos óptimos, que sugieren la necesidad de disponer de más datos para mejorar la eficacia de los modelos y conseguir mejores resultados.

Resulta interesante considerar la inversión en proyectos de este tipo ya que pueden contribuir a la detección temprana de enfermedades como el cáncer de mama, cuya incidencia está aumentando en España, ofreciendo un método de diagnóstico menos doloroso para los pacientes.



\section{Otros TFGs relacionados con el párkinson}
Otro ejemplo del uso de \textit{machine learning} para asistir, en este caso, a pacientes con párkinson, es el TFG realizado por un compañero de la carrera el curso pasado. Debido al acuerdo de colaboración entre la UBU y la Asociación Parkinson Burgos, se han llevado a cabo varios Trabajos de Fin de Grado relacionados con esta enfermedad. En este caso se aborda el proyecto de Catalin Andrei Cacuci, estudiante de Ingeniería Informática, quien presentó el curso pasado un TFG titulado <<Identificación de Parkinson por visión artificial>>\footnote{\url{https://github.com/cataand/tfg-paddel}}.

El proyecto se centró en la creación de un sistema capaz de detectar, mediante visión artificial, la presencia de bradicinesia (un síntoma presente en personas con párkinson que se manifiesta como ralentización del movimiento). Los individuos debían grabar un vídeo haciendo un movimiento de pinza con los dedos índice y pulgar, y el sistema detectaba cualquier alteración en el movimiento.

Este proyecto podría ser muy beneficioso para facilitar un primer diagnóstico, especialmente tras el aumento del uso de la telemedicina durante la pandemia.


Existen más Trabajos de Fin de Grado realizados en la UBU en colaboración con esta asociación, pero se destaca el realizado por Sara González Bárcena, estudiante de Ingeniería de la Salud, quien presentó el curso pasado un TFG titulado <<Detección de la actividad muscular de las personas con enfermedad de Parkinson>>
\footnote{\url{https://github.com/saragonzalezbarcena/TFG_Deteccion_Activ_Muscular}}.

Este proyecto se incluye en la sección de trabajos relacionados debido a que describe el desarrollo de un sensor diseñado para analizar la marcha de pacientes con párkinson, similar al sensor desarrollado por SENSE4CARE~\cite{sense4care} en 2020 que genera los archivos CSV de datos utilizados en este proyecto. Específicamente, el sensor desarrollado por la alumna es capaz de analizar la duración del ejercicio, el número de bloqueos durante el período de actividad y detectar desequilibrios entre ambos lados del cuerpo del paciente. Incorpora sensores inerciales (acelerómetro y giroscopio) para detectar la información.

Estos proyectos son muy importantes, ya que diseñan dispositivos muy específicos que, aunque son poco comunes y costosos, son muy útiles para ayudar al personal sanitario a diseñar terapias más precisas y personalizadas, mejorando la calidad de vida de los pacientes.



\section{Justificación de la necesidad del proyecto}
Después de exponer el contexto actual relacionado con los temas abordados en el proyecto, se procede a justificar resumidamente la necesidad de este.

Como se ha podido observar, las aplicaciones y plataformas web están a la orden del día. Son utilizadas diariamente por personas de todas las edades y en ámbitos de la vida muy variados. En el ámbito médico en específico, se empezaron a popularizar durante la cuarentena vivida en el año 2020, cuando las restricciones impidieron el acceso físico a los servicios de salud, lo que resultó en la detección tardía de enfermedades o monitorizaciones deficientes de las enfermedades de los pacientes, afectando gravemente a su salud. 

Herramientas como la desarrollada en este proyecto, que permiten a pacientes y médicos llevar un seguimiento de la enfermedad de forma remota, e incluso predecir la tendencia de la enfermedad a lo largo del tiempo, podrían haber sido de mucha utilidad tanto durante la pandemia como en la actualidad.

Aunque existen y se han mostrado aplicaciones disponibles actualmente en el mercado que ofrecen funcionalidades similares, la mayoría de ellas están disponibles solo en inglés y son de pago. Por lo tanto, existe una necesidad de desarrollar aplicaciones web como la de este proyecto, accesibles para pacientes de habla hispana, en concreto para pacientes con párkinson, y gratuitas, resultando asequibles para personas de cualquier nivel socio-económico. 